\documentclass[12pt,a4paper]{moderncv}
% style options are 'casual' (default), 'classic', 'oldstyle' and 'banking'
% color options 'blue' (default), 'orange', 'green', 'red', 'purple', 'grey' and 'black'
\moderncvtheme[purple]{classic}  
\usepackage[utf8]{inputenc}
\usepackage[scale=0.8]{geometry}
\usepackage{lmodern}% Latin Modern typeface (font)
\usepackage[T1]{fontenc}
\renewcommand{\familydefault}{\sfdefault}% Latin Modern's sans serif font family as default
\usepackage[french]{babel}
\usepackage{color}
\usepackage{tabularx}
\definecolor{gris}{gray}{0.3}
\def\arraystretch{1.5}%


\firstname{Carl-Philipp}
\familyname{Harmant \\ 
				\large{\textcolor{gris}{Software Engineer}}}
\title{Seeking a full time employment}              
\address{837 W Wrightwood Ave \#1}{Chicago, IL 60614}
\mobile{+1 (312) 513 6681}                    
\email{cp.harmant@gmail.com}  

\begin{document}
\maketitle

\section{Personal Synopsis}
Currently working as a software consultant, I am looking for a new challenge. I am passionate about building awesome software, and IT in general. I really like being part of a team and I think communication is very important. My preferred programming language is Java, I use it everyday at work and in my personal projects. At the moment, I am very interested in Android and Node.js, but I am open to anything. I obtained the OCJP certification two years ago in addition to my Master's degree in computer science. I am a very motivated and curious person.

\section{Professional Experience}
\cventry{2012-2013\\ \scriptsize{Feb. to Now}}
		{Software Consultant}
		{Cameleon Software}{Chicago}
		{United States}
		{As a consultant, I work for different customers and for several projects that use Cameleon technology: Java, web services, Oracle database, SalesForce.com, JBoss etc. I dealt wih the functional and the technical part of the projects.}{}

\cventry{2011\\ \scriptsize{Apr. to July}}
		{Java EE developer}
		{Fujitsu Technology Solutions}
		{Luxembourg}
		{Luxembourg}
		{In a Java EE context, I made a solution to improve the way to use Arquillian. Arquillian is a tool, made by JBoss, which allows developers to test the components of their application in a container. I made improvements to the test process by the creation of a Maven plugin which generates an Arquillian test class; and automatic search of dependencies needed for the archive deployed on the server. I had to utilise Maven, JUnit, EJBs, Cobertura, and many other tools and frameworks.}{}
				
\cventry{2010\\ \scriptsize{June to July}}
		{Analyst programmer}
		{Sailendra}
		{Nancy}
		{France}
		{During a project run by Sailendra for a pharmaceutical company, I was partnered with the project
manager. I had to work with the client to ascertain and identify their specific requirements.
Consequently, I had to learn Zope/Plone and then create a component (DTML and Python).}{}

\cventry{2010\\ \scriptsize{Jan. to March}}
		{Analyst programmer}
		{Sailendra}
		{Nancy}
		{France}
		{Sailendra created a technology of user profiling. That technology creates a profile of each user and predicts their comportments based on their actions and artificial intelligence algorithms.
My task was to create a Firefox plugin for their technology. I managed the project from initiation to implementation overseeing the planning, functional specifications, technical specifications, development of the plugin, development of the server, utilisation of Sailendra’s libraries, and testing. I had to learn new technologies and frameworks, for example, XUL and Spring (JEE). I was given complete autonomy for the project.}


\section{Skills / Languages}

\newcolumntype{F}{>{\setlength\hsize{0.3\hsize}}X}%
\newcolumntype{S}{>{\setlength\hsize{0.5\hsize}}X}%
\begin{tabularx}{1\textwidth}{ F S X }
    \textbf{Development} & Java, C\# & JEE, Android, Maven, Axis, Arquillian, Cobertura, EJB, Ibatis, SOAP, Hibernate, RMI, Spring, JSP, JSTL, Log4J, JUnit, API yahoo map, WFC, XNA, Subversion. And also C, Python, XUL.
 \\
   			& PHP, Flex, HTML5,\newline CSS3, JavaScript & I have created many websites following the W3C standards. Using of JQuery.
 \\
   			& IDE & Eclipse, Visual Studio and Adobe Flash Builder. \\
   			& Modeling & UML, Merise. \\
  		 \textbf{Database} & MySQL, Oracle,\newline SQL server,\newline MongoDB  & Successful completion of projects using those databases. \\
  		 \textbf{Server} & Apache, Tomcat,\newline Jboss & Installation, configuration and utilisation. \\
  		 \textbf{Oper. Sys.} & Linux, Windows &  \\
  		 \textbf{Languages} & \multicolumn{2}{l}{French: As native language.} \\
  			& \multicolumn{2}{l}{English: Fluent. TOEIC Score: 890.} \\
\end{tabularx}





\section{Education}
\cventry{2013}{Master of Software Engineering}{Exia.Cesi}{Nancy, France}{}{}
\cventry{2011}{Bachelor of Software Engineering}{Exia.Cesi}{Nancy, France}{}{}

\section{Certifications}

\cventry{2013}
		{MongoDB for Node.js Developers}
		{MongoDB University}
		{}
		{}
		{}

\cventry{2011}
		{OCJP: Oracle Certified Java SE 6 Programmer}
		{Oracle}
		{}
		{}
		{}

\section{Activities and Interests}
During my spare time, I enjoy sports such as running, biking, and competition basketball. I also like
discovering and learning about new technologies and products.\\
I enjoy traveling to foreign countries (USA, Europe, Caribbean, Australia) to experience different
languages, lifestyles and cultures.

\section{Social}
\cventry{LinkedIn}
		{\href{http://www.linkedin.com/in/carlphilipp}{http://www.linkedin.com/in/carlphilipp}}
		{}
		{}
		{}
		{}
\cventry{GitHub}
		{\href{https://www.github.com/carlphilipp}{https://www.github.com/carlphilipp}}
		{}
		{}
		{}
		{}
\end{document}

%\section{Referees}

%\newcolumntype{H}{>{\setlength\hsize{0.3\hsize}}X}%
%\begin{tabularx}{1\textwidth}{ H X }
%	Jérôme Cahuzac & Head of Professional Services at Cameleon Software, Chicago, US\newline
%						jcahuzac@cameleon-software.com\\
%	Régis Lhoste  & President of Sailendra, France\newline
%						regis.lhoste@sailendra.fr\\
%	Sébastien Guérin & Training Manager at Exia.Cesi, France\newline
%						sguerin@cesi.fr\\
%\end{tabularx}


\grid
